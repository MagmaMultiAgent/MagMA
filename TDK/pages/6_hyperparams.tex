\chapter{Hyperparameters}
\label{appx:hyperparams}

\section{Single Unit Testbench}
\label{app:a}

\begin{table}[htbp]
    \centering
    \begin{tabularx}{\textwidth}{|X|C{1cm}|C{1cm}|C{1cm}|C{1cm}|C{1.2cm}|C{1.3cm}|C{1.2cm}|C{1.3cm}|}
        \hline
        \textbf{Hyperparameter} & \textbf{ARS} & \textbf{A2C} & \textbf{DQN} & \textbf{PPO} & \textbf{QR-DQN} & \textbf{R-PPO} & \textbf{TRPO} & \textbf{MPPO} \\
        \hline
        \hline
        policy & MLP & MLP & MLP & MLP & MLP & LSTM-MLP & MLP & MLP \\
        \hline
        \hline
        learning\_rate & 2e-2 & 7e-4 & 3e-4 & 3e-4 & 7e-4 & 3e-4 & 1e-4 & 3e-4 \\
        \hline
        n\_steps & - & 50 & - & 50 & - & 50 & 50 & 50 \\
        \hline
        batch\_size & - & 1024 & 1024 & 1024 & 1024 & 1024 & 1024 & 1024 \\
        \hline
        n\_epochs & - & - & - & 10 & - & 10 & - & 10 \\
        \hline
        gamma & - & 0.99 & 0.99 & 0.99 & 0.99 & 0.99 & 0.99 & 0.99 \\
        \hline
        gae\_lambda & - & 0.95 & - & 0.95 & - & 0.95 & 0.95 & 0.95 \\
        \hline
        tau & - & - & 0.95 & - & 0.95 & - & - & - \\
        \hline
        ent\_coef & - & 1e-3 & - & 1e-3 & - & 01e-3 & - & 1e-3 \\
        \hline
        vf\_coef & - & 0.5 & - & 0.5 & - & 0.5 & - & 0.5 \\
        \hline
        exp\_fraction & - & - & 0.1 & - & 0.01 & - & - & - \\
        \hline
        exp\_initial\_eps & - & - & 1.0 & - & 1.0 & - & - & - \\
        \hline
        exp\_final\_eps & - & - & 0.05 & - & 0.1 & - & - & - \\
        \hline
        clip\_range & - & - & - & 0.2 & - & 0.2 & - & 0.2 \\
        \hline
        target\_kl & - & - & - & 0.05 & - & 0.05 & 0.05 & 0.05 \\
        \hline
        \hline
        net\_params & \begin{tabular}{@{}c@{}}128 \\ 128\end{tabular} & \begin{tabular}{@{}c@{}}128 \\ 128\end{tabular} & \begin{tabular}{@{}c@{}}128 \\ 128\end{tabular} & \begin{tabular}{@{}c@{}}128 \\ 128\end{tabular} & \begin{tabular}{@{}c@{}}128 \\ 128\end{tabular} & \begin{tabular}{@{}c@{}}128 \\ 128\end{tabular} & \begin{tabular}{@{}c@{}}128 \\ 128\end{tabular} & \begin{tabular}{@{}c@{}}128 \\ 128\end{tabular} \\
        \hline
    \end{tabularx}
    \captionsetup{justification=justified, singlelinecheck=false, width=1\linewidth, labelfont=bf} 
    \caption{The table displays key hyperparameters utilized in the single unit benchmark study (\autoref{sec:single-unit-testbench-results}). It's important to note that for the R-PPO algorithm, the value 128 also denotes the hidden size of the LSTM layer, whereas for the other algorithms, 128 only refers to the size of the linear projection layers.}
    \label{tab:single-agent-hyperparameters}
\end{table}

\section{Monolithic Approach}
\label{app:b}

\begin{table}[H]
    \begin{tabularx}{\textwidth}{|X|C{3.8cm}|C{3.8cm}|}
        \hline
        \multicolumn{1}{|Y|}{\textbf{Hyperparameter}} & \textbf{Monolithic Training} & \textbf{Comparison With Other Methods} \\
        \hline
        \hline
        policy & MLP & MLP \\
        \hline
        number of trainable parameters & X & X \\
        \hline
        shared actor and value embeddings & false & false \\
        \hline
        \hline
        algorithm & M-PPO & M-PPO \\
        \hline
        optimizer & Adam & Adam \\
        \hline
        weight decay & false & false \\
        \hline
        learning rate & 2e-4 & 2e-4 \\
        \hline
        max episode length & 256 & 1000 \\
        \hline
        parallel environments & 32 & 8 \\
        \hline
        rollout steps & 8,192 & 8,192 \\
        \hline
        mini-batch size & 1024 & 1024 \\
        \hline
        epochs per train cycle & 10 & 10 \\
        \hline
        train until N steps & 204,800 & 204,800 \\
        \hline
        gamma & 0.99 & 0.99 \\
        \hline
        GAE lambda & 0.95 & 0.95 \\
        \hline
        entropy coefficient & 1e-3 & 1e-3 \\
        \hline
        value coefficient & 0.5 & 0.5 \\
        \hline
        gradient normalization & true & true \\
        \hline
        maximum gradient norm & 0.5 & 0.5 \\
        \hline
        clip range & 0.2 & 0.2 \\
        \hline
        clip value loss & true & true \\
        \hline
        target KL & none & none \\
        \hline
        \hline
        runs per test & 3 & 3 \\
        \hline
        evaluation environments & 12 & 12 \\
        \hline
        evaluation after every N train cycles & 1 & 1 \\
        \hline
    \end{tabularx}
    \captionsetup{justification=justified, singlelinecheck=false, width=1\linewidth, labelfont=bf} 
    \caption{Table containing all key hyperparameters utilized in the monolithic approach tests. The \textbdd{Monolithic Training} hyperparameters were used in \autoref{sec:monolithic-approach}, \autoref{sec:monolithic-approach-results}.}
    \label{tab:mono-approach-hyperparameters}
\end{table}

\section{Hybrid Approach}
\label{app:c}

\begin{table}[H]
    \begin{tabularx}{\textwidth}{|X|C{3.8cm}|C{3.8cm}|}
        \hline
        \multicolumn{1}{|Y|}{\textbf{Hyperparameter}} & \textbf{Trajectory Separation Test} & \textbf{Comparison With Other Methods} \\
        \hline
        \hline
        policy & MLP & MLP \\
        \hline
        number of trainable parameters & 208,457 & 209,033 \\
        \hline
        shared actor and value embeddings & false & false \\
        \hline
        \hline
        algorithm & M-PPO & M-PPO \\
        \hline
        optimizer & Adam & Adam \\
        \hline
        weight decay & false & false \\
        \hline
        learning rate & 2e-4 & 2e-4 \\
        \hline
        max episode length & 256 & 1000 \\
        \hline
        parallel environments & 16 & 4 \\
        \hline
        rollout steps & 4,096 & 4,096 \\
        \hline
        mini-batch size & 512 & 512 \\
        \hline
        epochs per train cycle & 10 & 10 \\
        \hline
        train until N steps & 102,400 & 102,400 \\
        \hline
        gamma & 0.99 & 0.99 \\
        \hline
        GAE lambda & 0.95 & 0.95 \\
        \hline
        entropy coefficient & 1e-3 & 1e-3 \\
        \hline
        value coefficient & 0.5 & 0.5 \\
        \hline
        gradient normalization & true & true \\
        \hline
        maximum gradient norm & 0.5 & 0.5 \\
        \hline
        clip range & 0.2 & 0.2 \\
        \hline
        clip value loss & true & true \\
        \hline
        target KL & none & none \\
        \hline
        \hline
        runs per test & 3 & 3 \\
        \hline
        evaluation environments & 12 & 12 \\
        \hline
        evaluation after every N train cycles & 1 & 1 \\
        \hline
    \end{tabularx}
    \captionsetup{justification=justified, singlelinecheck=false, width=1\linewidth, labelfont=bf} 
    \caption{Table containing all key hyperparameters utilized in the hybrid approach tests. The \textbdd{Trajectory Separation Test} hyperparameters were used in \autoref{sec:trajectory-separation}, \autoref{subsec:trajec_reduc}, \autoref{subsec:methodcomp} and \autoref{subsec:ablation}, while  the "Comparison with Other Methods" hyperparameters were only used in \autoref{subsec:comparison}.}
    \label{tab:hybrid-approach-hyperparameters}
\end{table}
